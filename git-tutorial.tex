\documentclass{beamer}
\usepackage[utf8]{inputenc}
\usepackage[T1]{fontenc}
%\usepackage[french]{babel}
\usepackage{amsmath, amsfonts, amsthm, amssymb}
\usepackage{hyperref}

\title{GIT tutorial}
\author{Pauline Hubert and Nadia Lafrenière}
\date{FPSAC software days, July 2019}
\begin{document}
	\maketitle
	\begin{frame}{Motivation}
		Dossier de Duncan
		\pause
		PhD Comics
		\pause
		Long échange de courriels
	\end{frame}
	\begin{frame}{What software to use? Pros and cons}
		\begin{tabular}{lp{0.35\linewidth}p{0.35\linewidth}}
%			\hline
			& Pros & Cons\\
			\hline
			Email & "Easy" to use & So many!\\
			\hline
			Overleaf & Many people can work at the same time & Only online, latex only\\
			\hline
			Dropbox & Easy to use & No two people can work at the same time, no version control\\
			\hline
			Git & Version control, automatic fusion of modifications, any file type & More difficult to start\\
%			\hline
		\end{tabular}
	\end{frame}
	\begin{frame}{What is Git?}
		Git is a version control software. \newline
		
		\begin{itemize}
			\item It keeps track of every modifications.
			\item You can go back in time to a previous version.
			\item It allows many people to work in parallel.
		\end{itemize}	
	\end{frame}
	\begin{frame}{Online servers for Git}
		Store your data somewhere. \newline 
			\begin{itemize}
				\item Local : on your computer.
				\item Online servers : Github, Bitbucket, GitLab, University server, etc. \newline
			\end{itemize}
		If you use an online server, you will also have a local version on your computer. If you want to work with other people, you have to use a server. 
	\end{frame}
	\begin{frame}{Now that you are convinced... Installation}
		\begin{itemize}
			\item Linux : Type in shell \texttt{sudo apt install git}
			\item MacOS : Go to \url{git-scm.com/download/mac}
			\item Windows : Go to \url{gitforwindows.com}
		\end{itemize}
	\end{frame}
	\begin{frame}[fragile]{Configuration}
	To share code, you need to identify yourself:
	\begin{lstlisting}[language=bash,fontadjust]
$ git config --global user.name "John Doe"
$ git config --global user.email "john.doe@email.com"
	\end{lstlisting}
	\begin{verbatim}
		$ git config --global user.name "John Doe"
		$ git config --global user.email "john.doe@email.com"
	\end{verbatim}
	\end{frame}

	\begin{frame}{What's happening?}
		git status
	\end{frame}
	\begin{frame}{First steps: Creating a folder + initializing}
		To get code from a remote repository:
		\begin{verbatim}
		$ git clone https://address-of-the-repo.git
		\end{verbatim}
	\end{frame}
	\begin{frame}{Getting code : cloning}
		contenu...
	\end{frame}
	\begin{frame}{Modifications}
		
	\end{frame}
	\begin{frame}{Sharing modifications}
		contenu...
	\end{frame}
	\begin{frame}{Getting updated code \& solving conflicts}
	
	\end{frame}
	\begin{frame}
		DIFF (meld)
	\end{frame}
	\begin{frame}
		branches
	\end{frame}
	\begin{frame}
		Your turn
	\end{frame}	
\end{document}